\mcchap{Implementazione}{cap:cap4}
\section{Backend}
Il backend è la parte del sistema che si occupa dell'elaborazione dei dati e 
dell'interazione con il database. È responsabile della gestione delle 
richieste e delle risposte tra il frontend e il database.

Nel caso dell’Hotfix Response Center il backend è stato sviluppato 
utilizzando il framework Django, per Python. Django è stato utilizzato assieme 
alla libreria Django REST framework per realizzare le API REST necessarie per 
la comunicazione tra frontend e backend.


\subsection{Api rest}
Le API REST (Representational State Transfer) sono un tipo di architettura 
software utilizzata per la progettazione e lo sviluppo di servizi web. 
Le API REST consentono alle applicazioni di comunicare tra loro e 
scambiare dati utilizzando i protocolli standard del web, come HTTP. 
Le API REST sono basate su un insieme di principi fondamentali. 
In primo luogo, utilizzano le risorse come concetto centrale, dove ogni 
risorsa rappresenta un'entità specifica (ad esempio, un oggetto, un 
utente, una transazione) e viene identificata da un URL univoco.

Le API REST utilizzano metodi HTTP come GET, POST, PUT, PATCH e DELETE 
per consentire alle applicazioni di accedere, creare, modificare o 
eliminare le risorse. Ad esempio, una richiesta HTTP GET a un'API REST 
può essere utilizzata per ottenere i dettagli di una risorsa, mentre una 
richiesta POST può essere utilizzata per creare una nuova risorsa.
Le API REST utilizzano anche la rappresentazione dei dati, di solito nel 
formato JSON o XML, per trasmettere le informazioni tra client e server. 
Questo permette una facile interoperabilità tra diverse piattaforme e 
linguaggi di programmazione.

Uno dei vantaggi delle API REST è la loro scalabilità e flessibilità. 
Le API REST sono state progettate per essere stateless, il che significa 
che ogni richiesta deve contenere tutte le informazioni necessarie per 
essere elaborata e non richiede lo stato persistente del server. Ciò 
consente una maggiore scalabilità del sistema e la possibilità di 
distribuire le risorse su più server. ~\cite{wiki:api_rest}\\


\subsection{Django}
Django REST Framework, spesso abbreviato in DRF, è un framework 
open-source per lo sviluppo di API REST. Offre una serie di funzionalità 
che semplificano la creazione di API scalabili e ben documentate.
DRF si basa sui concetti fondamentali di Django, come il modello 
MVC (Model-View-Controller) e l'ORM (Object-Relational Mapping). 
Sfrutta la potenza di Django per fornire una struttura solida e coerente 
per la creazione di API REST. Con DRF, è possibile definire i modelli di 
dati utilizzando le classi Django (che verranno poi convertiti in tabelle). 
Una delle caratteristiche chiave di DRF è la sua gestione avanzata della 
serializzazione dei dati. Questo consente di convertire facilmente gli 
oggetti Python in formati serializzati come JSON o XML e viceversa. 
DRF offre anche un sistema completo per la gestione dell'autenticazione 
e dell'autorizzazione delle API. È possibile configurare facilmente 
l'autenticazione basata su token, l'autenticazione OAuth2, l'autenticazione 
basata su sessione e altre modalità di autenticazione. Inoltre, supporta la 
definizione di permessi personalizzati per controllare l'accesso agli 
endpoint delle API in base a regole specifiche.

Un'altra caratteristica utile di DRF è la gestione delle route API. 
DRF offre un sistema di routing che consente di definire 
facilmente gli endpoint delle API e associarli alle corrispondenti view. 
Questo semplifica l'organizzazione delle API in base a risorse specifiche 
e facilita la creazione di endpoint RESTful standardizzati.
Infine, DRF include un modulo di documentazione automatica che genera 
automaticamente la documentazione delle API basandosi sulle definizioni 
delle API e dei serializer. Questo rende più semplice per gli sviluppatori 
e gli utenti delle API comprendere e utilizzare correttamente le risorse 
fornite dall'applicazione. ~\cite{wiki:django}\\


\subsection{Esempio}
I seguenti pezzi di codice illustrano in Django la realizzazione di 
un'API REST per la gestione delle campagne all'interno del backend.
In particolare, l'esempio illustra come definire un modello di dati per una 
campagna, creare URL per gestire le richieste relative alle campagne, definire 
una vista che si occupa di recuperare e creare campagne utilizzando il modello 
e serializzare i dati della campagna utilizzando un serializer.

\lstinputlisting[style=custompython, language=Python]{code/django.py}

Nel file models.py viene definita la classe Campaign che estende la classe 
models.Model. Questa classe rappresenta il modello di dati per una campagna e 
contiene tutti i campi definiti nel paragrafo XXXXXXXXXXXXX relativo alla definizione 
delle tabelle. Ogni campo definisce il tipo di dati che può contenere e le 
eventuali opzioni di configurazione. 

Nel file views.py viene definita la classe CampaignListCreateAPIView che 
gestisce le richieste relative alle campagne. La variabile queryset definisce 
l'insieme di dati su cui operare, in questo caso tutte le campagne ordinate per 
l'ID. Vengono anche specificati i  filtri e l’ordinamento da utilizzare. 
La variabile serializer\_class definisce la classe del serializer da utilizzare 
per la serializzazione dei dati in JSON.

Nel file serializers.py viene definita la classe CampaignSerializer che gestisce 
la serializzazione dei dati della campagna. La classe Meta all'interno del 
serializer specifica il modello di riferimento (Campaign), i campi da includere 
nella serializzazione ('\_\_all\_\_' indica tutti i campi del modello) e i campi di 
sola lettura che non possono essere modificati.

Nel file urls.py viene definito il percorso delle API per gestire le richieste 
relative alle campagne. In particolare, quando viene inviata una richiesta all'endpoint 
/campaigns/, viene utilizzata la classe CampaignListCreateAPIView come vista 
per gestire la richiesta. Questa vista estende la classe generics.ListCreateAPIView 
fornita da Django REST Framework e fornisce funzionalità per visualizzare 
l'elenco delle campagne e creare una nuova campagna.\\

Facendo richieste di HTTP GET sull’endpoint /campaigns/ verrà fornito l’elenco 
delle campagne inserite nel database. Sarà possibile utilizzare i filtri per 
ricercare tra tutte le campagne solo quelle desiderate. Le informazioni saranno 
restituite in formato JSON con i campi definiti nel serializer.

Facendo invece una richiesta di HTTP POST sullo stesso endpoint sarà possibile 
creare una nuova campagna. Le informazioni della nuova campagna andranno 
inserite nel body della richiesta.
Per entrambe le richieste dovrà essere fornito il token di autenticazione per 
verificare l’identità e le autorizzazioni dell’utente che sta utilizzando i 
servizi.


\section{Frontend}
Il frontend è la parte dell’applicazione che interagisce direttamente con 
l’operatore. È responsabile della presentazione dei dati consentendo agli 
utenti di visualizzare e interagire con il contenuto dell'applicazione web.

Nel contesto dello sviluppo web, il frontend si riferisce alla combinazione 
di tecnologie, linguaggi di programmazione e strumenti utilizzati per creare 
l'interfaccia utente di un'applicazione web. Ciò include l'utilizzo di HTML 
(Hypertext Markup Language) per definire la struttura e il contenuto della 
pagina, CSS (Cascading Style Sheets) per definire il layout e l'aspetto 
visivo della pagina, e JavaScript per aggiungere interattività e funzionalità 
dinamiche.

In questo caso per lo sviluppo dell’interfaccia frontend si è deciso di 
utilizzare il framework Vue.js per semplificare lo sviluppo e migliorare 
l'efficienza di programmazione.

\subsection{Vue.js}
Vue.js è un framework JavaScript open-source progressivo utilizzato per la creazione di 
interfacce utente dinamiche e reattive. È incentrato sulla creazione di 
applicazioni web a pagina singola (Single-Page Applications) e si concentra 
sulla visualizzazione dei dati e sulla gestione dello stato dell'applicazione.

Le caratteristiche chiave di Vue.js includono la reattività dei componenti, 
la gestione dichiarativa degli elementi del documento HTML, un sistema di 
componenti componibile e la capacità di gestire lo stato dell'applicazione in 
modo efficiente. Vue.js offre anche una vasta gamma di strumenti e librerie di 
supporto, oltre a una comunità attiva che contribuisce allo sviluppo e 
all'evoluzione del framework.

Con la sua sintassi intuitiva e la facilità d'uso, Vue.js è diventato popolare 
tra gli sviluppatori per la creazione di interfacce utente complesse, ma 
mantenendo una curva di apprendimento accessibile. È ampiamente utilizzato per 
la creazione di applicazioni web moderne e reattive, fornendo un'esperienza 
utente ottimizzata e una manutenzione semplificata del codice. 
~\cite{wiki:vuejs}\\


\subsection{Esempio}
todo: CODICE DELIVERY HUB creazione campagna.\\

Questa schermata del frontend viene utilizzata per creare una nuova campagna.
L’operatore, al termine dell’inserimento di tutte le informazioni richieste, 
schiaccera il bottone “Crea campagna” per salvare la campagna.
Il codice JavaScript associato al click del bottone farà una chiamata POST 
alla API del backend con la quale passerà tutte le informazioni inserite 
dell’utente e verrà salvata la campagna all’interno del database.

