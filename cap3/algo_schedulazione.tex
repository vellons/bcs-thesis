%%%%%%%%%% CAPITOLO DI TESI %%%%%%%%%%
%
% Capitolo "3" algoritmo di schedulazione
%
%%%%%%%%%%%%%%%%%%%%%%%%%%%%%%%%%%%%%%
\section{Algoritmo di schedulazione}
L’Hotfix Response Center è un tool che verrà utilizzato in momenti 
critici, per aggiornare da remoto centinaia di server vulnerabili, 
in modo automatico.

Ricopre quindi un ruolo molto importante l’algoritmo di schedulazione. 
L’algoritmo dovrà essere in grado di trovare una schedulazione 
ottimale per aggiornare i server tenendo conto, durante l'elaborazione, 
di numerosi fattori, dettagliati in seguito. L’affidabilità e la 
precisione dell’algoritmo sarà fondamentale per poter garantire che 
tutto avvenga in modo automatico e nelle corrette tempistiche.\\

La schedulazione avverrà generalmente su un periodo temporale 
di 2/3 settimane. Ogni giorno della settimana verrà suddiviso in 48 
slot da 30 minuti. Per ogni slot di tempo sarà possibile aggiornare 
massimo 5 server contemporaneamente. Questo rende possibile aggiornare 
fino a 240 server al giorno. La scelta di posizionare più server nello 
stesso slot dovrà avvenire solamente nel caso non ci siano altri slot 
liberi, preferendo sempre slot vuoti. In una situazione ideale si schedula 
solo 1 server per slot, questo per garantire uniformità nel calendario.\\

\textbf{Dati in input:} data e ora di inizio e fine campagna, elenco dei 
server da schedulare dove per ogni server sono presenti:
\begin{itemize}
\item atlantis\_id: id del gestionale dal quale provengono le informazioni del server;
\item fascia di manutenzione/aggiornamento;
\item gruppo di patch: duo o più server con lo stesso gruppo di patch non possono 
essere schedulati insieme;
\item dipendenze tra server. Questi server devono essere i primi ad essere 
schedulati a distanza di 24 ore.
\end{itemize}

\textbf{Dati in output:} schedulazione per ogni server in modo da poterli 
aggiornare il prima possibile.


\subsection{Concetto di euristica}
L’algoritmo realizzato per la schedulazione si basa su un principio euristico. 
Un'euristica è una tecnica o un metodo basato sull'esperienza o sulla 
conoscenza generale del problema, che cerca di trovare una soluzione 
ragionevole in tempi ragionevoli, anche in presenza di problemi di 
complessità elevata. In altre parole, un'euristica è un approccio basato 
sul buon senso, sulla logica e sulla creatività che, seppur non garantisca la 
soluzione ottimale del problema, può fornire risultati accettabili in tempi 
ragionevoli. Le euristiche sono spesso utilizzate per risolvere problemi di 
ottimizzazione in cui il tempo di calcolo richiesto per trovare una soluzione 
ottimale è troppo elevato.\\

Nel caso in questione l’algoritmo di schedulazione non esplora tutti i possibili 
casi per trovare la soluzione migliore ma si basa su dei calcoli preliminari 
per decidere quali server schedulare per primi.
Questa tecnica permette di ridurre drasticamente il tempo di esecuzione 
dell’algoritmo ottenendo un tempo di esecuzione lineare e fornendo una 
soluzione ottimale per creare un calendario di schedulazione per l’aggiornamento.\\

L’euristica si baserà sul calcolo dell’“availability score” che verrà utilizzato 
per decidere con che ordine schedulare i server all’interno del calendario.

È appropriato utilizzare un’euristica in quanto il calcolo della soluzione 
ottimale, che controllerebbe tutte le possibili schedulazioni, richiederebbe un 
tempo di esecuzione quadratico. Il calcolo e l’utilizzo del “availability score” 
per la schedulazione richiederebbe una complessità lineare in termini di tempo e 
spazio che fornirebbe una soluzione accettabile nel minor tempo possibile.


\subsection{Fascia di manutenzione}
Anche chiamata fascia di aggiornamento se presente indica le fasce orarie 
in cui il server può essere aggiornato e riavviato. 
Ad esempio: MAR-MER-GIO(13:00-18:00) indica che un server può essere 
aggiornato martedì, mercoledì e giovedì dalle 13 alle 18.
Se la fascia di manutenzione non è presente il server può sempre essere aggiornato.

La fascia di manutenzione è rappresentata da una stringa che deve essere 
interpretata correttamente prima di poter eseguire l’algoritmo di schedulazione. 

I giorni della settimana sono rappresentati dalle stringhe: 
LUN, MAR, MER, GIO, VEN, SAB, DOM. Seguiti dalla fascia oraria tra parentesi.
Una fascia di manutenzione può anche essere composta da più blocchi di fasce 
orarie, concatenati dal carattere +, come in questo esempio: 
LUN-MAR-MER(20:00-24:00)+LUN-MAR-MER(00:00-05:00)+SAB-DOM(00:00-24:00)\\

Per poter interpretare correttamente la stringa e capire durante la settimana in quali 
slot di 30 minuti può essere inserito il server è stata creata una classe 
Python capace di comprendere il significato della stringa.

Le funzionalità principali della classe sono:
\begin{itemize}
\item Interpretare la stringa della fascia di manutenzione;
\item Gestire una struttura dati per memorizzare in quali slot durante la 
settimana il server può essere in manutenzione;
\item Esporre un metodo che comunica se il server è in manutenzione in un 
determinato slot. Questo metodo verrà poi utilizzato dall’algoritmo di schedulazione.
\end{itemize}

La classe interpreta la stringa utilizzando delle regex (abbreviazione di 
"regular expressions", ovvero "espressioni regolari" in italiano) in grado di 
riconoscere pattern di testo ed eseguire successivamente gli opportuni calcoli.
La struttura dati per memorizzare la disponibilità degli slot si basa su un 
dizionario in cui la chiave del dizionario è la rappresentazione a stringa 
dell’inizio dello slot in secondi.

\textbf{Esempio:}\\
Chiave 0 = primo slot, lunedì mattina alle ore 00:00\\
Chiave 1800 = secondo slot, lunedì mattina alle ore 00:30\\
Chiave 86400 = slot del martedì alle ore 00:00\\

Questi valori sono ottenuti impostando la costante RESOLUTION\_IN\_MINUTE = 30. 
Quindi durante la settimana, nel dizionario, saranno presenti 336 slot da 30 minuti.

Dopo aver inizializzato la struttura dati, per salvare le disponibilità negli 
slot, inizia l’interpretazione della stringa della fascia di manutenzione.\\

Di seguito è riportato il metodo calc\_available\_slots presente all'interno della classe MaintenanceWindowParser:

\lstinputlisting[style=custompython, language=Python]{code/calc_available_slot.py}

Al termine dell’esecuzione il dizionario contenuto in self.data conterrà 336 chiavi. 
Ad ogni chiave è associato un valore booleano. Se il valore corrisponde a True il 
server è disponibile in quello slot per l’aggiornamento.


\subsection{Calcolo dello score}
Dopo aver trovato per ogni server quali sono gli slot disponibili durante la 
settimana si procede con il calcolo dell’“availability score” di ogni server. 
Questo valore numerico sarà utilizzato per decidere quali server andranno schedulati per primi.

L'Availability score corrisponde al numero di secondi in cui il server è disponibile 
per l’aggiornamento durante la settimana.
Più lo score sarà basso e meno slot ha un server a disposizione per essere aggiornato.\\

Attraverso l’approccio euristico è stato constatato che schedulando per primi i 
server con un availability score più basso non si compromette la schedulazione degli 
altri server. Non risulterà quindi necessario provare ogni possibile schedulazione 
per trovare la migliore ma ci si potrà accontentare della schedulazione deterministica 
che verrà creata assegnando prima i server con score più basso negli slot in cui 
possono essere aggiornati.
I server che saranno schedulati per ultimi saranno quelli con score più alto.
Questo significa che tali server potranno probabilmente essere schedulati 
negli slot lasciati liberi dai server con score più basso.\\


Di seguito è riportato il metodo get\_num\_of\_seconds\_of\_available\_slot della 
classe MaintenanceWindowParser:

\lstinputlisting[style=custompython, language=Python]{code/get_num_of_seconds_of_available_slot.py}


\subsection{Albero delle dipendenze}
Prima dell’esecuzione dell’algoritmo di schedulazione dovrà essere creato 
l’albero delle dipendenze, per i server con dipendenze, cioè che necessitano 
che un server venga aggiornato prima di eseguire l’aggiornamento del server 
stesso. Questi server saranno schedulati per primi, tenendo una distanza di 24 
ore tra uno slot e l’altro.

La struttura utilizzata è un albero. Alla radice è presente un nodo padre. 
I figli di primo livello sono tutti i server che non hanno dipendenze ma che 
fanno dipendere altri server.
I figli di secondo livello sono i server che dipendono dai server di primo livello
che dovranno essere aggiornati 24 ore dopo l'aggiornamento con successo dei filgi 
di primo livello, e così via\dots

L’albero viene gestito da una semplice classe Python, che viene popolata durante 
lo scorrimento preliminare di tutti i server.


\subsection{Esecuzione dell’algoritmo}
L’algoritmo di schedulazione risultante è abbastanza semplice. 
Si inizia con il calcolare il priority score di ogni server.
Questo valore è la somma tra l’availability score e il grado che 
ha il server all'interno dell’albero delle dipendenze, se i server non 
hanno dipendenze avranno grado 0.

Poi si ordinano i server per priority score crescente (riga 7).
Successivamente, nel ciclo for di riga 14, si scorrono tutti i server 
da schedulare presenti nella campagna.
Per ogni server si estrae la lista di tutti gli slot per cui provare 
la compatibilità. Questi slot da provare vengono restituiti dal metodo 
get\_list\_of\_slot\_to\_try (riga 19). Questo metodo restituisce gli slot 
in ordine temporale restituendo però prima gli slot che hanno meno server 
già assegnati.
Per ogni slot restituito si controlla, se è compatibile con la fascia di 
manutenzione del server (riga 21). Se la fascia di manutenzione permette 
l’utilizzo di quello slot il server viene schedulato altrimenti si prova 
per tutti gli slot rimasti.

Nel caso non ci sia nessuno slot disponibile, per effettuare 
l’aggiornamento del server, questo non avrà una schedulazione e non sarà 
inserito a calendario (riga 31). In questo caso l’operatore, prima 
dell’avvio della campagna, dovrà selezionare manualmente uno slot per 
schedulare il server.\\

Codice parziale dell'algoritmo di schedulazione:
\lstinputlisting[style=custompython, language=Python]{code/scheduling.py}

CALCOLARE COMPLESSITA'

ESEMPIO TEMPORALI DI ESECUZIONE
