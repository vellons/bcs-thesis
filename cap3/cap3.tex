\mcchap{Progettazione}{cap:cap3}
\section{Entità ed associazioni}
Nella fase di progettazione concettuale del tool, uno dei principali
obiettivi è stata la definizione di un modello di dati che potesse 
supportare le funzionalità principali dell'applicazione.
A tal fine, si è deciso di realizzare un database relazionale che 
contenesse diverse tabelle. La creazione delle tabelle è gestita 
tramite Django, un framework Python molto conosciuto nello sviluppo di 
applicativi backend, verrà dettagliato in seguito.\\

\noindent Il database relazionale comprende quindi diverse tabelle, tra cui:
\begin{itemize}
\item campaign: la tabella principale del database, che rappresenta una 
campagna di aggiornamento. Ogni record in questa tabella contiene 
informazioni sulle specifiche della campagna;
\item item: tabella che contiene l'insieme dei server che devono essere 
aggiornati, correlati alla campagna tramite una relazione uno-a-molti. 
Ogni record in questa tabella rappresenta un singolo server da aggiornare;
\item updatefile: tabella che contiene le informazioni sui file di aggiornamento 
effettivi che verranno installati sui server selezionati. 
Anche questa tabella è correlata alla tabella “campaign” tramite una 
relazione uno-a-molti e contiene un record per ogni diverso sistema 
operativo dei server selezionati in una campagna di aggiornamento;
\item updatejob: tabella che contiene le informazioni sulle sessioni 
di aggiornamento avviate su un server. Ogni record in questa tabella 
rappresenta un'operazione di aggiornamento. È possibile avere più operazioni 
di aggiornamento per un singolo server, per consentire di ripetere 
l’aggiornamento in caso di fallimento;
\item updatejobstatushistory: tabella che tiene traccia dello stato d'avanzamento 
dell’operazione di aggiornamento. Questa tabella è utilizzata 
per tenere traccia degli stati di ogni singola operazione di aggiornamento 
presente nella tabella.
\end{itemize}

\begin{figure}[H]
  \begin{flushright}
    \centering
    \includegraphics[width=0.95\textwidth]{imgs/ER_schema.png}
    \caption{Schema entità-relazione del database relazionale}
    \label{fig:Schema entità-relazione del database relazionale}
  \end{flushright}
\end{figure}

Come mostra lo schema ER della figura
~\ref{fig:Schema entità-relazione del database relazionale}, le tabelle del database sono state 
progettate in modo da riflettere le relazioni tra le entità coinvolte 
nel sistema.

La tabella “item” è correlata alla tabella “campaign” attraverso una 
relazione uno-a-molti, in quanto ad ogni campagna di aggiornamento 
corrisponde un insieme di server che devono essere aggiornati.

La tabella “updatefile” è sempre correlata alla tabella “campaign” 
attraverso una relazione uno-a-molti, in quanto ad ogni campagna di 
aggiornamento corrisponde un insieme di file che 
devono essere installati sui server selezionati.

La tabella “updatejob” è correlata alla tabella “item” in quanto ogni 
operazione di aggiornamento è associata ad un singolo server.

La tabella “updatefile” è associata alla tabella “campaign”, sempre 
attraverso relazioni uno-a-molti, perché in base al sistema operativo 
ogni server della campagna avrà il suo file di aggiornamento. 

La tabella “updatejobstatushistory” è correlata alla tabella 
“updatejob” tramite una relazione uno-a-molti, in quanto ad ogni 
operazione di aggiornamento corrisponde un insieme di stati di 
avanzamento dell'operazione.

Inoltre, nello schema, è presente la tabella auth\_user. 
Questa tabella è utilizzata da Django per gestire le utenze che 
utilizzano il backend. Infatti, ogni campagna creata è associata ad 
un creatore e ogni file di aggiornamento caricato è associato ad un 
utente che l’ha caricato.

\subsection{Definizioni delle tabelle}

\noindent \textbf{Campaign:}
\begin{itemize}
\item id: l'ID univoco della campagna;
\item name: il nome associato alla campagna;
\item status: lo stato della campagna, può essere draft (bozza), scheduled (pianificata), started (avviata) o closed (chiusa);
\item type: il tipo della campagna, può essere pkg (pacchetto) o cve;
\item mode: la modalità della campagna, può essere standard o zero day;
\item id\_ticket: l'ID del ticket associato alla campagna;
\item start\_at: la data e l'ora di inizio della campagna;
\item end\_at: la data e l'ora di fine della campagna;
\item idrs: l’ID del cliente a cui è associata la campagna;
\item created\_by: l'utente che ha creato la campagna;
\item algorithm\_started\_at: la data e l'ora di inizio dell'algoritmo di elaborazione;
\item algorithm\_scheduling\_note: una nota di pianificazione dell'algoritmo di elaborazione;
\item created\_at: la data e l'ora di creazione della campagna;
\item updated\_at: la data e l'ora di ultima modifica della campagna.\\
\end{itemize}

\noindent \textbf{Item:}
\begin{itemize}
\item id: l'ID univoco dell’item;
\item campaign\_id: l’ID della campagna associata all'elemento;
\item atlantis\_id: l'ID del gestionale CMDB sul quale sono salvate le informazioni del server;
\item scheduled\_at: la data e l'ora di pianificazione dell'aggiornamento;
\item has\_winrm: indica se il server ha WinRM abilitato, attraverso un check eseguito alla creazione della campagna;
\item has\_credential: indica se il server ha una credenziale associata, attraverso un check eseguito alla creazione della campagna;
\item approved: indica se l'elemento è stato approvato automaticamente (se need\_approve è False);
\item need\_approve: indica se l'elemento richiede approvazione;
\item manually\_update: indica se l'elemento deve essere aggiornato manualmente;
\item approved\_at: la data e l'ora di approvazione dell'elemento;
\item approved\_by: l'utente che ha approvato l'elemento.
\end{itemize}

\noindent \textbf{UpdateFile:}
\begin{itemize}
\item id: l'ID del file.
\item campaign\_id: l'ID della campagna associata all'aggiornamento;
\item os\_name: il codice del sistema operativo associato all'aggiornamento;
\item kb\_code: il codice univoco dell'aggiornamento di sicurezza;
\item execute\_params: i parametri di esecuzione dell'eseguibile di aggiornamento;
\item path: l'url del percorso del bucket su cui è stato caricato il file di aggiornamento;
\item updated\_by: l'utente che ha caricato il file di aggiornamento;
\item s3\_version: la versione di S3 del file di aggiornamento;
\item created\_at: la data e l'ora di creazione del file;
\item updated\_at: la data e l'ora di ultima modifica del file.
\end{itemize}

\noindent \textbf{UpdateJob:}
\begin{itemize}
\item id: campo UUID che rappresenta la chiave univoca della sessione di aggiornamento;
\item item\_id: l'ID che si riferisce al modello Item e specifica l'elemento di cui viene effettuato l'aggiornamento;
\item provision\_id: contiene l'ID del gestore dei task schedulati remoti, utilizzato per far partire l'aggiornamento da remoto;
\item token\_sha256: contiene il valore SHA256 del token utilizzato per inviare aggiornamenti di stato da remoto;
\item ticket\_id: campo che può contenere l'ID del ticket che viene aperto in caso di problemi di aggiornamento automatico;
\item check\_status: campo che contiene lo stato di tutti i controlli di monitoraggio, prima di avviare l'aggiornamento;
\item silencer\_id: l'ID del downtime associato al server per evitare segnalazioni dovute al riavvio della macchina durante l'aggiornamento;
\item created\_at: la data e l'ora dell'avvio dell'aggiornamento.
\end{itemize}

\noindent \textbf{UpdateJobStatusHistory:}
\begin{itemize}
\item update\_job\_id:  l'ID della sessione di aggiornamento;
\item status: specifica lo stato dell'aggiornamento. I possibili valori sono: “Ready”, 
“Execute error from provision”, “Waiting for ping”, “Timeout ping”, 
“Waiting for download process”, “Timeout download”, “Running”, “Timeout running”, “Rebooting”, 
“Failed”, “Completed”;
\item created\_at: la data e l'ora del passaggio di stato;
\item note: campo che contiene eventuali note aggiuntive per lo stato dell'aggiornamento.
\end{itemize}


\section{Design architettura}
L’Hotfix Response Center è stato realizzato con un’architettura a container. 
I container sono ambienti isolati e leggeri che includono tutto il necessario 
per eseguire un'applicazione, compresi il codice, le dipendenze e le librerie.

\subsection{Kubernetes}
La gestione dei container è affidata a Kubernetes, un sistema open-source per 
l'orchestrazione dei container. È progettato per automatizzare la distribuzione, 
la scalabilità e la gestione delle applicazioni containerizzate.

Kubernetes fornisce un'infrastruttura per coordinare e gestire il deployment 
dei container in modo efficiente. Si può definire come le applicazioni dovrebbero 
essere eseguite, specificando requisiti di risorse, definendo relazioni tra i 
componenti e gestendo le dipendenze.

Su Kubernetes, i container sono organizzati e gestiti all'interno di unità 
chiamate “pod”. Un pod è il più piccolo oggetto di gestione in Kubernetes e 
rappresenta un singolo insieme di uno o più container, con uno spazio di 
archiviazione e una rete condivisa.\\

\noindent Le principali caratteristiche di  Kubernetes sono:
\begin{itemize}
\item Orchestrazione: coordina e distribuisce i container su un cluster di macchine (insieme di server). Gestisce automaticamente il posizionamento dei container, il bilanciamento del carico, la scalabilità e il ripristino in caso di errori;
\item Autoscaling: supporta l'autoscaling automatico in base al carico dell'applicazione. Può aumentare o diminuire il numero di repliche dei container in base alle metriche definite, assicurando che l'applicazione abbia sempre le risorse necessarie per funzionare correttamente;
\item Gestione delle risorse: consente di definire i requisiti di risorse per i container, come CPU e memoria, e li alloca in modo appropriato all'interno del cluster. Ciò garantisce che le risorse siano utilizzate in modo efficiente e che le applicazioni non interferiscano tra loro;
\item Rolling updates e rollback: facilita gli aggiornamenti delle applicazioni senza interruzioni. Si possono eseguire aggiornamenti dei container in modo graduale e controllato, monitorando lo stato e, se necessario, effettuando il rollback a una versione precedente.
\end{itemize}

Le risorse dell’Hotfix Response Center sono tutte state inserite in un namespace 
dedicato (una sorta di contenitore virtuale) in modo da poter coesistere con 
altri microservizi all'interno del cluster.\\

\noindent All’interno del namespace dedicato sono presenti le seguenti risorse:
\begin{itemize}
\item Deployment frontend: esegue un pod con il web server nginx che mette a disposizione i file statici del front-end, relativo al pannello di amministrazione;
\item Deployment backend: esegue tre pod che consentono l’esecuzione del backend in Django;
\item Cronjob di avvio: schedulato ogni 30 minuti. Questo job seleziona tutte le campagne attive e fa partire gli aggiornamenti per i server schedulati in quello slot temporale;
\item Cronjob di monitoraggio: schedulato ogni 10 minuti. Questo job monitora gli aggiornamenti in corso. Deve controllare che gli aggiornamenti automatici procedano senza problemi rilevando se si verificano dei timeout.
\end{itemize}

La figura ~\ref{fig:Design architettura} 
mostra uno schema rappresentativo dell’architettura.

\begin{figure}[H]
  \begin{flushright}
    \centering
    \includegraphics[width=0.95\textwidth]{imgs/architecture_design.pdf}
    \caption{Design architettura}
    \label{fig:Design architettura}
  \end{flushright}
\end{figure}

Il rettangolo azzurro rappresenta il namasepace dedicato all'interno del cluster.
Il backend contatta anche altri microservizi, presenti in altri namespace, dello stesso cluster.
Ad esempio contatta le API del gestionale per recuperare le informazioni dei server da aggiornare.

\label{subsec:Ansible}
\subsection{Ansible}
Un altro componente importante dell’architettura è Ansible, una piattaforma 
open-source per l'automazione IT. Ansible viene utilizzato per semplificare 
la gestione e l'automazione di sistemi IT complessi, utilizzando un 
linguaggio semplice basato su YAML chiamato “Playbook” per 
definire le azioni da eseguire sui nodi di destinazione.\\

Ansible viene utilizzato per connettersi da remoto ai server dei clienti per 
eseguire l’aggiornamento in modo automatico. Tramite Ansible viene lanciato un 
playbook sul nodo di controllo che, in base al sistema operativo del server da 
aggiornare, lancia un eseguibile per Windows o per Linux.
Questo eseguibile si occuperà dell'aggiornamento remoto del server. 
L’eseguibile, in modo autonomo, si preoccuperà di scaricare dallo storage 
S3 il file di aggiornamento e di installarlo sul server. 
Ad ogni passo d’esecuzione l’eseguibile comunica al backend dell’Hotfix Response Center, 
tramite le API, i passi dell’aggiornamento, seguendo lo schema di 
figura ~\ref{fig:Diagramma a stati dell'aggiornamento di un server}.

 \begin{figure}[H]
  \begin{flushright}
    \centering
    \includegraphics[width=0.90\textwidth]{imgs/update_statues.png}
    \caption{Diagramma a stati dell'aggiornamento di un server}
    \label{fig:Diagramma a stati dell'aggiornamento di un server}
  \end{flushright}
\end{figure}

Se l’aggiornamento avviene con successo il server andrà in stato “completed”.
Nel caso qualcosa non dovesse funzionare correttamente il cronjob di monitoraggio 
rileverà il timeout d’esecuzione e marcherà l’aggiornamento come “failed”.
