%%%%%%%%%% CAPITOLO DI TESI %%%%%%%%%%
%
% Capitolo "1" Capitolo 1
%
%%%%%%%%%%%%%%%%%%%%%%%%%%%%%%%%%%%%%%
\mcchap{Sguardo generale sul sistema}{cap:cap1}
\section{Le vulnerabilità nei software}
Il mondo attuale è caratterizzato da un livello sempre più elevato di
digitalizzazione, per questo i sistemi informatici regolano non solo
le reti digitali, ma anche quelle più tradizionali, come gli impianti di
produzione e distribuzione d’energia, le reti di trasporto e molto altro. 
Le infrastrutture informatiche costituiscono quindi l’anima dell’operatività 
nazionale e aziendale.

Queste infrastrutture informatiche sono sempre più frequentemente oggetto 
di attacchi hacker da parte di individui ed organizzazioni. 
Attacchi che in alcuni casi sfruttano vulnerabilità software presenti nei 
sistemi, in altri casi sfruttano, con l’inganno, individui poco pratici in 
ambito informatico, ad esempio tramite il phishing, per accedere alle reti 
aziendali.

Per vulnerabilità si intende una debolezza nel software che, se sfruttata, 
viola almeno uno dei principi di sicurezza: confidenzialità, integrità e 
disponibilità. Le vulnerabilità software possono essere di diversa natura,
ad esempio possono essere presenti nel codice sorgente del software, oppure 
possono essere presenti nel software stesso, oppure possono essere presenti 
nel software che viene utilizzato per la gestione del software stesso, 
ad esempio i sistemi operativi.


\section{Concetti per la comprensione delle vulnerabilità}
Svariati team di sicurezza scoprono e divulgano pubblicamente le vulnerabilità
in modo indipendente. Dal 1999 il MITRE, un'organizzazione no profit 
americana, finanziata dal Dipartimento di Sicurezza Nazionale 
Statunitense (NSA), ha introdotto il programma CVE.

CVE è un acronimo che sta per Common Vulnerabilities and Exposures
(vulnerabilità comuni ed esposizioni) e si tratta di un database pubblico nel 
quale vengono aggiunte e aggiornate vulnerabilità, in modo che chiunque possa 
accedervi e utilizzarlo. 
È uno strumento molto utile e viene utilizzato come standard da vari istituti 
di ricerca nel mondo. Ogni vulnerabilità è rappresentata da un identificatore 
CVE unico, rappresentato nel formato: “CVE-anno-numero”. 
Ad esempio: CVE-2021-44228 (nel caso della famosa vulnerabilità 
di Apache Log4j2).

Ogni vulnerabilità inserita nel CVE ha dei parametri di classificazione. 
Uno dei più importanti è il CVSS ovvero Common Vulnerability Scoring System 
cioè uno standard che indica la gravità di una vulnerabilità informatica 
da 0 a 10, dove 10 indica il livello di vulnerabilità più critico. 
Due usi comuni del CVSS sono il calcolo della gravità delle vulnerabilità 
scoperte sui propri sistemi e come fattore di prioritizzazione delle 
attività di riparazione delle vulnerabilità. Il National Vulnerability 
Database (NVD) fornisce punteggi CVSS per quasi tutte le vulnerabilità.

Viene definita zero-day una vulnerabilità non pubblicamente nota, che può 
essere utilizzata dai cracker per attaccare un sistema, attraverso un 
exploit (software che sfrutta la vulnerabilità per bucare il sistema). 
Vengono chiamate zero-day proprio perché gli sviluppatori hanno “zero giorni” 
per riparare la falla nel software, prima che qualcuno la possa sfruttare.
Nel momento in cui il bug viene risolto, la zero-day perde la sua importanza
perché non può più essere usata contro quel sistema.
La pericolosità di questi tipi di vulnerabilità sta proprio nel fatto che i 
produttori del software non hanno ancora rilasciato una patch di sicurezza 
per risolvere il problema, o nel peggiore dei casi non ne sono 
neanche a conoscenza. 
Questi tipi di vulnerabilità sono le più ricercate nei software e spesso 
vengono anche rivendute al mercato nero (deep web), per farci dei soldi.
Molto spesso però sono le stesse aziende, sviluppatrici del software, ad avere 
un programma di bug bounty, per identificare nuove vulnerabilità nei propri
sistemi, in cambio di premi in denaro per l’utente che segnala a loro
le vulnerabilità scoperte presenti nel loro software.



\section{ALTRO}
\section{Citazioni}
Per citare bisogna editare il file biblio/biblio.bib e aggiungere in formato bibtex la citazione poi citarla così: citazione~\cite{STORY}. 
Automaticamente verrà aggiunta alla bibliografia quando citata.

\section{Tabelle}
Le tabelle si citano come le immagini: Tab.~\ref{tab:mlp}

\begin {table}[H]
\caption {Confronto MLP e SVM in media} \label{tab:mlp} 
\begin{center}
\begin{tabular}{|c|c|c|}
  
  \hline
  \rowcolor[gray]{.6}
  \textbf{SVM jaccard} & \textbf{MLP jaccard} & \textbf{MLP jaccard – SVM jaccard} \\
  
  \hline
  \rowcolor[gray]{.8}
  0,4275 & 0,29172 & \textcolor{red}{-0,13578}\\
  \hline
\end{tabular} 
\end{center}
\end{table}

\section{code}

Il codice può essere scritto inline List.~\ref{code:bash}:
\begin{lstlisting}[language=DebianBash, style=basic, label=code:bash, caption=sample bash]
git clone https://gitlab.com/nicolalandro/ai_block.git
\end{lstlisting}

Oppure caricato da file:
\lstinputlisting[style=custompython, language=Python]{code/iris_keras.py}
