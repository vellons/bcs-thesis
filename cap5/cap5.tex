\mcchap{Conclusioni}{cap:cap5}

\section{Soluzioni già presenti sul mercato}
L'Hotfix Response Center è uno strumento realizzato da zero per adattarsi ad un 
contesto aziendale in cui ci sono già altri sistemi di patching in funzione. 
L'obiettivo era creare un tool indipendente dai sistemi aziendali già in uso, 
che potesse funzionare in modo semplice e soprattutto rapido in quanti più 
scenari possibili.\\

Esistono però diverse soluzioni simili ma più complete sul mercato. Un esempio è 
BigFix di IBM, un software che permette di gestire in maniera centralizzata 
le patch di sicurezza e le configurazioni di sistema di tutti i computer di una 
rete aziendale.
La soluzione BigFix di IBM si basa su un'architettura client-server in cui i 
dispositivi sono dotati di un agente che comunica costantemente con il server 
centrale. Questo permette alle organizzazioni di eseguire azioni di gestione e 
sicurezza su larga scala, come l'applicazione di patch, la distribuzione di software, 
la configurazione dei dispositivi e molto altro, in modo centralizzato e 
automatizzato. ~\cite{ibm:bigfix}

Sebbene BigFix offra funzionalità di gestione e sicurezza avanzate, è importante 
considerare alcuni fattori prima di adottarlo. In primo luogo, bisogna tenere 
presente che BigFix richiede l'acquisto di una licenza per ogni macchina gestita, 
il che può comportare costi significativi per le organizzazioni con un gran numero 
di server da gestire. Questo aspetto finanziario potrebbe influenzare la decisione 
di adottare BigFix come soluzione per la gestione delle sole patch.
Inoltre, per funzionare, BigFix ha bisogno di un suo agente funzionante sui server. 
L’agente è un applicativo da installare su ogni server che si vuole controllare. 
Il suo scopo è mantenere la comunicazione tra server e console centrale, in attesa 
di ricevere azioni da eseguire sul server gestito. L’installazione dell’agente può 
richiedere un impegno aggiuntivo in termini di tempo e risorse.

Soluzioni come BigFix offrono una vasta gamma di funzionalità di gestione e sicurezza, 
è però importante valutare attentamente i costi fissi di tale soluzione, le 
implementazioni e le esigenze specifiche dell'organizzazione.\\ 

L'Hotfix Response Center si propone di fornire una soluzione flessibile e semplice 
per la gestione delle patch senza però avere costi di gestione. L’Hotfix è stato 
progettato per essere indipendente dai sistemi aziendali già in uso e per funzionare 
in modo semplice e rapido in diversi scenari aziendali, solamente quando necessario. 
La sua flessibilità consente di adattarlo alle esigenze specifiche dell'organizzazione 
senza dover installare o mantenere nessun agente sul server che si vuole aggiornare.\\

In conclusione, mentre soluzioni come BigFix offrono una vasta gamma di funzionalità di 
gestione e sicurezza, è importante valutare attentamente i costi, le implementazioni e 
le esigenze specifiche dell'organizzazione prima di prendere una decisione. 
L'Hotfix Response Center si propone di fornire una soluzione, semplice e veloce per 
l’installazione degli aggiornamenti in situazioni di emergenza, in cui la rapidità è 
fondamentale. 
L’Hotfix Response Center può essere usato su tutti i server su cui è 
disponibile l’accesso remoto senza che sia necessario alcun software già installato 
sui server che si vuole aggiornare.

\section{Sviluppi futuri}
L’Hotfix Response Center rappresenta attualmente una soluzione efficace per la gestione in 
modo rapido delle campagne di aggiornamento, per risolvere le vulnerabilità. Tuttavia, per 
rendere il processo ancora più efficiente e automatizzato, sono in programma sviluppi futuri 
che introdurranno funzionalità avanzate.

Uno degli aspetti chiave in fase di sviluppo è la creazione automatica delle nuove campagne 
di aggiornamento. 
Attualmente, l'avvio delle campagne richiede un'azione manuale da parte degli operatori, che 
devono selezionare tutti i server impattati e caricare i file di aggiornamento. L'obiettivo 
è implementare una funzionalità che suggerisca automaticamente la creazione di nuove campagne 
in base alle informazioni raccolte da altri software interni all’azienda che recuperano 
l’elenco di tutti i software installati sui server.
Gli applicativi in questione monitorano costantemente i software installati sui server e 
inviano ad un database centrale tutte le informazioni recuperate.
Utilizzando le informazioni rese pubbliche dal NIST, all’interno del National Vulnerability 
Database, è possibile effettuare dei controlli di analisi proattiva per individuare se tra i 
software installati nei server dei clienti sono installati software con delle vulnerabilità.

Attraverso questo controllo incrociato l’Hotfix Response Center può suggerire agli 
amministratori di sistema la creazione di campagne per risolvere automaticamente le 
vulnerabilità presenti nei server impattati dalle vulnerabilità più gravi.\\

L'implementazione di questa funzionalità consentirà di automatizzare il processo di 
individuazione e gestione delle vulnerabilità, semplificando ulteriormente il lavoro degli 
amministratori di sistema e riducendo i tempi di risposta agli aggiornamenti critici. Ciò 
garantirà un ambiente più sicuro e protetto per le infrastrutture gestite.

Inoltre, l'Hotfix Response Center continuerà a essere sviluppato per integrarsi con altre 
piattaforme e strumenti utilizzati all'interno dell'azienda. L'obiettivo è creare un ecosistema 
completo che permetta una gestione centralizzata degli aggiornamenti, sfruttando 
al massimo le sinergie con i sistemi esistenti.\\

In conclusione, l'Hotfix Response Center è un progetto in continua evoluzione che mira a offrire 
soluzioni sempre più avanzate per la gestione delle campagne di aggiornamento. Grazie alle 
funzionalità future in fase di sviluppo, sarà possibile automatizzare il processo di 
individuazione e gestione delle vulnerabilità, migliorando la sicurezza e l'efficienza 
complessiva delle infrastrutture monitorate.
