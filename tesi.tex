%%%%%%%%%%%%%%%%%%%%%%%%%%%%%%%%%%%%%%%%%%%%%%%%%
%
% Università degli Studi dell'Insubria
% Tesi di Laurea triennale in Informatica
% Alex Vellone
%
%%%%%%%%%%%%%%%%%%%%%%%%%%%%%%%%%%%%%%%%%%%%%%%%%

%%\documentclass[a4paper, 11pt, twoside]{book} %% Two side for book
\documentclass[a4paper, 11pt, oneside]{book}
\usepackage[a-2b]{pdfx}
\usepackage[a4paper]{geometry}
\usepackage{lmodern}
\usepackage[italian]{babel}
\usepackage{textcomp}
\usepackage{xcolor}
\usepackage{url}
\usepackage{amsfonts}
\usepackage{float}
\usepackage{booktabs}
\usepackage{longtable}
\usepackage{makeidx}
\usepackage{fancyhdr}
\usepackage[times]{quotchap}
\usepackage{multirow}
\usepackage{version}

\usepackage{listings}
\usepackage{color}
\usepackage{xcolor}

\definecolor{gray}{rgb}{0.9,0.9,0.9}

%%modifiche per il codice
\lstdefinestyle{basic}{  
  basicstyle=\footnotesize\ttfamily,
  numbers=left,
  numberstyle=\tiny\color{orange}\ttfamily,
  numbersep=5pt,
  backgroundcolor=\color{white},
  showspaces=false,
  showstringspaces=false,
  showtabs=false,
  frame=single,
  rulecolor=\color{black},
  captionpos=b,
  keywordstyle=\color{blue}\bf,
  commentstyle=\color{gray},
  stringstyle=\color{green},
  keywordstyle={[2]\color{red}\bf},
}


\lstdefinelanguage{DebianBash}{
  morekeywords={cd, apt-get, &&, time, curl, sudo, echo, cat, echo, chmod,
    sleep, done, git , pip},
  morecomment=[l]{\#},
  morestring=[b]",
  alsodigit={-},
  alsoletter={&}
}

\lstdefinestyle{customjava}{
  language=Java,
  frame=tlrb,
  aboveskip=3mm,
  belowskip=6mm,
  showstringspaces=false,
  columns=flexible,
  basicstyle={\small\ttfamily},
  numbers=left,
  numberstyle=\tiny\color{orange}\ttfamily,
  numbersep=5pt,
  keywordstyle=\color{purple},
  commentstyle=\color{orange},
  stringstyle=\color{blue},
  breaklines=true,
  breakatwhitespace=true
  tabsize=3
}

\lstdefinestyle{custompython}{
  language=Python,
  frame=tlrb,
  aboveskip=3mm,
  belowskip=6mm,
  showstringspaces=false,
  columns=flexible,
  basicstyle={\small\ttfamily},
  numbers=left,
  numberstyle=\tiny\color{orange}\ttfamily,
  numbersep=5pt,
  keywordstyle=\color{purple},
  commentstyle=\color{orange},
  stringstyle=\color{blue},
  breaklines=true,
  breakatwhitespace=true
  tabsize=3
}


%% Aggiunge una linea al di sotto di ogni sezione principale
\usepackage[calcwidth]{titlesec}
\titleformat{\section}[hang]{\sffamily\bfseries}
 {\Large\thesection}{12pt}{\Large}[{\titlerule[0.4pt]}]

 \linespread{1.1}

%% Gestisce la grafica a seconda che si usi latex o pdflatex
\newif\ifpdf
\ifx\pdfoutput\undefined
\pdffalse % no pdflatex
\else
\pdfoutput=1 % pdflatex
\pdftrue
\fi
%
\ifpdf

\usepackage[pdftex]{floatflt,graphicx}
\DeclareGraphicsExtensions{.pdf,.mps,.png,.jpg}
\usepackage[pdftex]{hyperref}
\else
\usepackage{floatflt,graphicx}
\DeclareGraphicsExtensions{.eps}
\fi
\usepackage{subfigure}

\usepackage{algorithm}
\usepackage{algorithmic}
\usepackage[utf8]{inputenc} % per accenti

%%tabella con linee colorate
\usepackage{colortbl}


%%%%%%%%%%%%% NUOVI COMANDI E IMPOSTAZIONI %%%%%%%%%%%
\newenvironment{mcquote}
  {\begin{list}{}{
      \setlength{\rightmargin}{\leftmargin}}
         \item[]``\ignorespaces}
  {\unskip''\end{list}}
  
\newcommand{\mcchap}[2]{\protect{
 \chapter{#1}
 \label{#2}
}}

%% Gestione header: no header sulle dispari bianche
\makeatletter
\def\cleardoublepage{\clearpage\if@twoside \ifodd\c@page\else%
  \hbox{}%
  \thispagestyle{empty}%              % Empty header styles
  \newpage%
  \if@twocolumn\hbox{}\newpage\fi\fi\fi}
\makeatother

%% Link colorati e non riquadrati
%\hypersetup{
%    colorlinks,
%    linkcolor={red!50!black},
%    citecolor={blue!50!black},
%    urlcolor={blue!80!black}
%}

%% Link non colorati e non riquadrati
\hypersetup{hidelinks}

\newcommand{\codice}[1]{\protect\texttt{\small{#1}}}

\newcommand{\mcproc}[1]{\ensuremath{\mbox{\sc #1}}}

\newcommand{\mctodo}[1]{\protect{
  \bigskip
  \begin{tabular}{|p{13cm}} \textcolor{red}{\underline{TODO:}} \small{#1} \end{tabular}}
}

\newcommand{\mcnota}[1]{\protect{
  \bigskip
  \begin{tabular}{|p{13cm}} \underline{NOTA:} \small{#1} \end{tabular}}
}


%%%%%%%%%%%%%%%% METADATI DOCUMENTO %%%%%%%%%%%%%%%%%%
\date{}
\pdfinfo{
   /Author (Alex Vellone)
   /Title  (Strumento per installazione rapida e remota di aggiornamenti di sicurezza sui server)
   /PdfTitle (Strumento per installazione rapida e remota di aggiornamenti di sicurezza sui server)
   /CreationDate (D:20230525000500)  %format D:YYYYMMDDhhmmss
   /Subject (Hotfix Response Center)
   /Keywords (Hotfix,Vulnerability,CVE,Security,Tool)
}

%%%%%%%%%%%%%%%%%% INIZIO DOCUMENTO %%%%%%%%%%%%%%%%%%
\begin{document}
  \pagestyle{empty}

  %% Pagina del titolo
  \begin{titlepage}
  \begin{center}
    \begin{large}
      {\fontsize{20}{18}\selectfont\vspace*{0.50cm}Universit\`a degli Studi dell'Insubria}\\
      Dipartimento di Scienze Teoriche e Applicate (DiSTA)\\
      Corso di Laurea Triennale in Informatica
    \end{large}

    \vspace{1cm}
    \begin{figure}[h]
      \begin{center}
        \includegraphics[scale=0.25]{copertina/logounivector.pdf}
      \end{center}
    \end{figure}

      {
        \fontsize{26}{26}\usefont{OT1}{phv}{c}{n}\selectfont\par\vspace*{0.75cm}
        Tool per l'installazione rapida\\
        \vspace{.15em}di patch di sicurezza
      }
    \par

    \vfill
    \vspace{3cm}
    \begin{large}
      Relatore: Dott. Alberto Trombetta

      \vspace{1.0cm}
      Tesi di Laurea di\\
      Alex Vellone\\
      Matricola 741527\\
      \vspace{0.5cm}

    \end{large}

    Anno Accademico 2021-2022

  \end{center}
\end{titlepage}


  %% Dedica
  \frontmatter{}
  \begin{flushright}
    \vspace*{2cm}
    \textit{ 
    Ai miei genitori
    }
    \vspace{2.5cm}
  \end{flushright}
  \cleardoublepage

  %% Indice ed elenchi
  \pagenumbering{roman}
  \setcounter{page}{1}
  \setcounter{tocdepth}{2}
  \tableofcontents
  \listoffigures
  \listoftables

  %% Inizio capitoli
  \mainmatter{}

  %% Capitoli
  \pagestyle{fancy}
  \renewcommand{\chaptermark}[1]{\markboth{#1}{}} 
  \renewcommand{\sectionmark}[1]{\markright{\thesection\ #1}} 
  \fancyhf{} % delete current setting for header and footer 
  \fancyhead[LE,RO]{\bfseries\thepage} 
  \fancyhead[LO]{\bfseries\rightmark} 
  \fancyhead[RE]{\bfseries\leftmark} 
  \renewcommand{\headrulewidth}{0.8pt} 
  \renewcommand{\footrulewidth}{0pt} 
  \setlength{\headheight}{13.59999pt}
  \addtolength{\headheight}{0.5pt} % make space for the rule 
  \fancypagestyle{plain}{
    \fancyhead{} % get rid of headers on plain pages
    \fancyfoot[C]{\bfseries \thepage}
    \renewcommand{\headrulewidth}{0pt} % and the line 
  } 

  \cleardoublepage{}
  \setcounter{page}{1}
  \mcchap{Sguardo generale sul sistema}{cap:cap1}
\section{Le vulnerabilità nei software}
Il mondo attuale è caratterizzato da un livello sempre più elevato di
digitalizzazione, per questo i sistemi informatici regolano non solo
le reti digitali, ma anche quelle più tradizionali, come gli impianti di
produzione e distribuzione d’energia, le reti di trasporto e molto altro. 
Le infrastrutture informatiche costituiscono quindi l’anima dell’operatività 
nazionale e aziendale.

Queste infrastrutture informatiche sono sempre più frequentemente oggetto 
di attacchi hacker da parte di individui ed organizzazioni. 
Attacchi che in alcuni casi sfruttano vulnerabilità software presenti nei 
sistemi, in altri casi sfruttano, con l’inganno, individui poco pratici in 
ambito informatico, ad esempio tramite il phishing, per accedere alle reti 
aziendali.

Per vulnerabilità si intende una debolezza nel software che, se sfruttata, 
viola almeno uno dei principi di sicurezza: confidenzialità, integrità e 
disponibilità. Le vulnerabilità software possono essere di diversa natura,
ad esempio possono essere presenti nel codice sorgente del software, oppure 
possono essere presenti nel software stesso, oppure possono essere presenti 
nel software che viene utilizzato per la gestione del software stesso, 
ad esempio i sistemi operativi.


\section{Concetti per la comprensione delle vulnerabilità}
Svariati team di sicurezza scoprono e divulgano pubblicamente le vulnerabilità
in modo indipendente. Dal 1999 il MITRE, un'organizzazione no profit 
americana, finanziata dal Dipartimento di Sicurezza Nazionale 
Statunitense (NSA), ha introdotto il programma CVE.

CVE è un acronimo che sta per Common Vulnerabilities and Exposures
(vulnerabilità comuni ed esposizioni) e si tratta di un database pubblico nel 
quale vengono aggiunte e aggiornate vulnerabilità, in modo che chiunque possa 
accedervi e utilizzarlo. 
È uno strumento molto utile e viene utilizzato come standard da vari istituti 
di ricerca nel mondo. Ogni vulnerabilità è rappresentata da un identificatore 
CVE unico, rappresentato nel formato: “CVE-anno-numero”. 
Ad esempio: CVE-2021-44228 (nel caso della famosa vulnerabilità 
di Apache Log4j2).

Ogni vulnerabilità inserita nel CVE ha dei parametri di classificazione. 
Uno dei più importanti è il CVSS ovvero Common Vulnerability Scoring System 
cioè uno standard che indica la gravità di una vulnerabilità informatica 
da 0 a 10, dove 10 indica il livello di vulnerabilità più critico. 
Due usi comuni del CVSS sono il calcolo della gravità delle vulnerabilità 
scoperte sui propri sistemi e come fattore di prioritizzazione delle 
attività di riparazione delle vulnerabilità. Il National Vulnerability 
Database (NVD) fornisce punteggi CVSS per quasi tutte le vulnerabilità.\\

Viene definita zero-day una vulnerabilità non pubblicamente nota, che può 
essere utilizzata dai cracker per attaccare un sistema, attraverso un 
exploit (software che sfrutta la vulnerabilità per bucare il sistema). 
Vengono chiamate zero-day proprio perché gli sviluppatori hanno “zero giorni” 
per riparare la falla nel software, prima che qualcuno la possa sfruttare.
Nel momento in cui il bug viene risolto, la zero-day perde la sua importanza
perché non può più essere usata contro quel sistema. ~\cite{wiki:zero-day}
La pericolosità di questi tipi di vulnerabilità sta proprio nel fatto che i 
produttori del software non hanno ancora rilasciato un aggiornamento di sicurezza 
(patch) per risolvere il problema, o nel peggiore dei casi non ne sono 
neanche a conoscenza. 
Questi tipi di vulnerabilità sono le più ricercate nei software e spesso 
vengono anche rivendute al mercato nero (deep web), per farci dei soldi.
Molto spesso però sono le stesse aziende, sviluppatrici del software, ad avere 
un programma di bug bounty, per identificare nuove vulnerabilità nei propri
sistemi, in cambio di premi in denaro per l’utente che segnala a loro
le vulnerabilità scoperte presenti nel loro software.

  %%%%%%%%%% CAPITOLO DI TESI %%%%%%%%%%
%
% Capitolo "2" Capitolo 2
%
%%%%%%%%%%%%%%%%%%%%%%%%%%%%%%%%%%%%%%
\mcchap{Analisi del sistema}{cap:cap2}
\section{Scopo del progetto}
L'obiettivo di questo progetto di tesi è quello di creare uno strumento 
per l'installazione rapida di patch di sicurezza in grado di sanare, 
nel più breve tempo possibile, vulnerabilità di livello critico sui 
sistemi operativi dei server.
Questo tool verrà quindi utilizzato dopo che una determinata 
vulnerabilità viene scoperta dal produttore del software (vendor), e dopo che 
il produttore avrà rilasciato una patch di sicurezza per risolvere la 
vulnerabilità.
\begin{figure}[H]
  \begin{flushright}
    \centering
    \includegraphics[width=0.90\textwidth]{imgs/vulnerability_windows.png}
    \caption{Sequenza degli eventi di una vulnerabilità}
    \label{fig:Sequenza degli eventi di una vulnerabilità}
  \end{flushright}
\end{figure}

In una tipica sequenza degli eventi di una vulnerabilità, le patch di 
sicurezza si installano a partire dal punto temporale tp 
(mostrato nella figura ~\ref{fig:Sequenza degli eventi di una vulnerabilità}), 
momento in cui il produttore rilascia al pubblico la risoluzione alla 
falla di sicurezza e inizia la campagna di aggiornamento, per risolvere la criticità. 
Durante questo periodo di patching (aggiornamento), che può durare anche 
mesi, il sistema, nella maggior parte dei casi, è ancora vulnerabile e 
solo una rapida installazione, della patch di sicurezza, può mettere il 
sistema informatico al riparo da un attacco che sfrutta quella falla.

Ed è proprio per installare rapidamente le patch di sicurezza che nasce 
l’Hotfix Response Center, nome ufficiale di questo progetto di tesi.
Lo scopo di questo strumento è quindi quello di gestire, in automatico, 
il patching straordinario di più server Linux o Windows Server che sono 
impattati dalla falla di sicurezza, per il quale è stata rilasciata 
la patch.
L’installazione delle patch di sicurezza avviene tramite la creazione 
di campagne di aggiornamento. Durante la creazione della 
campagna andranno specificati tutti i server impattati e dovranno essere 
caricate, all’interno del tool Hotfix, per ogni sistema operativo, le 
patch per risolvere le vulnerabilità, che il produttore avrà provveduto 
a rilasciare pubblicamente.

Questo strumento è indipendente da qualsiasi altro strumento ed è composto
da un’interfaccia grafica (frontend), da dei servizi REST (backend), da
un database dedicato, da uno storage per il caricamento delle patch e da
due script necessari all’installazione della patch di sicurezza, uno 
specifico per Windows, scritto in Powershell, e uno per Linux, scritto 
in bash.


\section{Funzionalità principali}
\subsection{Creazione di una campagna}
Una campagna di aggiornamento è un insieme di server, con diverso sistema 
operativo, che devono essere aggiornati, entro un periodo che va tra due e 
quattro settimane. 

Per creare una campagna occorre decidere:
\begin{itemize}
\item data e ora di inizio campagna;
\item data e ora di fine campagna;
\item elenco dei server impattati;
\item patch di sicurezza (rilasciate dal produttore del software) per 
ogni sistema operativo;
\end{itemize}
L’interfaccia grafica (frontend) per la creazione delle campagne sarà 
divisa in 3 step:
\begin{itemize}
\item Inserimento delle proprietà della campagna, tra cui: nome, data di 
inizio e fine, email del responsabile e tipologia della campagna 
(dettagliate più avanti).
\item Selezione dei server impattati: tramite le API (Application Program 
Interface) del gestionale in cui sono salvate tutte le informazioni dei 
server, si selezionano tutti i server impattati. L’interfaccia dovrà 
permettere di filtrare l’elenco completo dei server in base a: nome del
server, sistema operativo, indirizzo IP e tag associati ai server. 
Per tutti i server selezionati saranno poi recuperate tutte le informazioni
per consentire l’aggiornamento (dettagliate più avanti).
\item Caricamento delle patch: dopo aver selezionato tutti i server impattati,
per ogni sistema operativo selezionato, si dovrà caricare la patch di sicurezza 
da installare sui server.
\end{itemize}

Per ogni server da aggiornare bisogna recuperare, tramite le API del gestionale 
esterno, le seguenti proprietà:
\begin{itemize}
\item nome;
\item sistema operativo in uso;
\item indirizzo ip;
\item utenza di amministrazione per la connessione remota, utilizzata per 
installare le patch;
\item fascia di manutenzione: se presente indica le fasce orarie in cui il
server può essere aggiornato e riavviato. Ad esempio: MAR-MER-GIO(13:00-18:00)
indica che un server può essere aggiornato martedì, mercoledì e 
giovedì dalle 13 alle 18;
\item gruppo di patch: se presente impone che due o più server dello
stesso gruppo di patch non siano mai sotto manutenzione contemporaneamente. 
Questo per evitare problemi su applicazioni distribuite in cui tutti i server 
vengono aggiornati contemporaneamente. Chiamato con il campo restart group.
\item dipendenze: alcuni server devono essere aggiornati prima di altri. 
Ad esempio: i server di quality vanno aggiornati prima dei server di produzione.
Se un server ha una dipendenza bisogna fare in modo che durante l’aggiornamento
venga rispettata, garantendo almeno 24 ore tra una aggiornamento e l'altro.
\end{itemize}
Le credenziali per la connessione al server fornite devono essere funzionanti e con 
permessi di amministratore sulla macchina. Queste credenziali verranno usate
per la connessione remota al server.
Per consentire l’aggiornamento i server Windows devono avere WinRM abilitato.
Mentre i server Linux devono avere SSH attivo. WinRM è un protocollo di 
gestione usato da Windows per comunicare in remoto con un server. 
Si tratta di un protocollo che comunica su HTTPS, ed è incluso in tutti 
i recenti sistemi operativi Windows. Da Windows Server 2012, WinRM è 
abilitato di default. 
SSH, o Secure Shell, è un protocollo di amministrazione remota che consente 
agli utilizzatori di creare una connessione sicura con i server Linux.
WinRM e SSH verranno utilizzati attraverso Ansible per l’esecuzione di task
remoti. Il funzionamento di Ansible verrà approfondito a 
pagina ~\pageref{subsec:Ansible}.

Le campagne possono avere due tipologie:
\begin{itemize}
\item standard: in cui vengono rispettate le fasce di manutenzione impostare 
per ogni server;
\item zero day: in cui tutti i server vengono schedulati per essere 
aggiornati il prima possibile.
\end{itemize}


\subsection{Schedulazione intelligente dei server}
Durante la creazione della campagna, quando si selezionano i server impattati 
dalla vulnerabilità di sicurezza, è possibile selezionare anche tutti i server
inseriti nel gestionale. 
È quindi necessario creare un algoritmo di schedulazione efficiente, capace di 
distribuire tutti i server da aggiornare, durante il periodo della campagna.
Tutte le campagne di aggiornamento durano tra due e quattro settimane. 
Ogni giorno viene diviso in più slot di 30 minuti. L’algoritmo di schedulazione 
dovrà essere in grado di trovare una schedulazione ottimale in base ai 
seguenti parametri:
\begin{itemize}
\item fascia di manutenzione corretta;
\item dipendenze tra server;
\item schedulare in modo da non avere due server con lo stesso 
“gruppo di patch” nello stesso slot;
\item schedulare in modo da avere massimo di server per slot, preferendo 
sempre slot vuoti. In una situazione ideale si schedula solo 1 server per slot.
\end{itemize}
L’algoritmo di schedulazione dovrà essere eseguito appena l’operatore, tramite 
l’interfaccia grafica, avrà inserito tutti i server da aggiornare nella 
campagna. L’algoritmo ha quindi pochi secondi per reperire le informazioni 
necessarie dei server e trovare la schedulazione migliore per la campagna.
Al termine dell’elaborazione verrà mostrato all’operatore un calendario con
tutte le schedulazioni di tutti i server nella campagna.
Attraverso il calendario della campagna dovrà essere possibile cambiare la 
data e/o l’ora di schedulazione, nel caso l’operatore ritenga opportuno
cambiare alcune schedulazioni.


\subsection{Caricamento file di aggiornamento}
Dopo l’inserimento di tutti i server vulnerabili, all'interno della campagna 
di aggiornamento, ci sarà un schermata di caricamento dei file di aggiornamento.
In questa schermata, per ogni differente sistema operativo dei server
selezionati, verrà dinamicamente mostrato l’elenco dei sistemi operativi
e un’interfaccia di caricamento per l’eseguibile/script di aggiornamento, 
da lanciare sulla macchina.
I file che saranno caricati dall’operatore saranno salvati in uno storage 
S3 dedicato.
Terminato il caricamento delle patch, per ogni sistema operativo, la campagna 
di aggiornamento può essere attivata dall’operatore.


\subsection{Avvio aggiornamento automatico}
La gestione degli aggiornamenti dei server dovrà essere automatica. 
Gli automatismi saranno gestiti con due cron jobs. I cron jobs sono 
dei programmi che vengono eseguiti ad orari prestabiliti. Cron è un programma di
utilità dei sistemi di tipo Unix che consente agli utenti di gestire 
le pianificazione di attività ripetute in un momento specifico.
La gestione automatica degli aggiornamenti sarà quindi gestita con 
dei task programmati che faranno partire l’aggiornamento sul server remoto.\\

Saranno presenti due cron job:
\begin{itemize}
\item job di avvio, schedulato ogni 30 minuti. Questo job seleziona tutte 
le campagne attive e farà partire gli aggiornamenti per i server schedulati 
in quello slot temporale. Ad esempio il job che parte alle 13:30 farà 
partire l’aggiornamento per i server schedulati dalle 13:30 alle 13:59; 
\item job di monitoraggio, schedulato ogni 10 minuti. Questo job monitorerà gli 
aggiornamenti in corso. Deve controllare che gli aggiornamenti automatici
procedano senza problemi rilevando se si verificano dei timeout 
(l’aggiornamento sta durando troppo oppure qualcosa si è interrotto 
senza preavviso). Questo job deve inoltre monitorare il corretto riavvio 
dei server dopo l’aggiornamento, per considerare l’aggiornamento 
come completato.
\end{itemize}


\subsection{Gestore dell’aggiornamento}
Un’ultima parte molto importante di questo sistema è lo script che gestirà 
l’aggiornamento sulla macchina remota. Ci saranno due diversi script: 
uno per i server Linux e uno per Windows server.
Lo script per Linux sarà realizzato in Bash mentre lo script per Windows 
server sarà realizzato in Powershell. Entrambi sono linguaggi di 
programmazione adatti per creare script iterativi da eseguire nella 
console dei rispettivi sistemi.
Il compito di questo script sarà quello di gestire l'aggiornamento del 
server remoto in modo automatico, comunicando con il server centrale lo 
stato dell’aggiornamento.
Lo scopo è quindi quello di scaricare la patch di aggiornamento dallo 
storage centrale e far partire l’aggiornamento sul server.
Per avere un maggiore controllo sullo stato di avanzamento lo script 
comunicherà gli stati intermedi durante l’operazione.\\

Gli stati che si susseguiranno durante l’operazione saranno:
\begin{itemize}
\item ping: test connessione all’avvio dello script; 
\item download patch: scaricamento dell’aggiornamento sul server;
\item esecuzione: installazione dell’aggiornamento;
\item reboot: riavvio del server, se necessario.
\end{itemize}
Al termine del riavvio il job di monitoraggio assegnerà lo stato 
completato. Se trascorre troppo tempo in uno step intermedio 
l’installazione andrà in timeout e l’aggiornamento sarà fallito.
La figura di pagina ~\pageref{fig:Diagramma a stati dell'aggiornamento di un server}
mostra il diagramma degli stati dell'aggiornamento.

\section{Diagramma delle attività}
La figura ~\ref{fig:Diagramma delle attività per la creazione di una campagna} 
rappresenta il diagramma delle attività per la creazione di una campagna di Hotfix.\\

Le entità coinvolte nella creazione sono:
\begin{itemize}
\item NIST (National Institute of Standards and Technology);
\item Il vendor del sistema operativo cioè colui che ha creato e gestisce il 
sistema operativo (Esempio Microsoft è il vendor di Windows);
\item Il tool Hotfix Response Center;
\item Il cliente.
\end{itemize}

Il flusso che porta alla creazione di una campagna di installazione 
parte dal NIST, l’ente americano che si occupa di gestire il National 
Vulnerability Database (NVD) fornendo punteggi CVSS per tutte le 
vulnerabilità note. Per quasi tutte le vulnerabilità, i vendor si preoccupano 
di fornire un patch, per sanare la vulnerabilità.
Il tool Hotfix Response Center è stato pensato per intervenire in situazioni 
d'emergenza, per patchare nel più breve tempo possibile le vulnerabilità critiche. 
Nel caso il CVSS (Common Vulnerability Scoring System) superi lo score di 9 su 
10 si controlla se tra i server dei clienti è presente la vulnerabilità appena scoperta.

Se i server dei clienti sono impattati si informa il cliente e, nel caso autorizzi 
l’aggiornamento straordinario, si procede con la creazione della campagna di 
aggiornamento per i server impattati dalla vulnerabilità.

Durante la creazione della campagna vengono caricate le patch di aggiornamento 
all’interno del tool e l’algoritmo di schedulazione cerca di trovare una 
schedulazione ottimale per patchare i server, nel più breve tempo possibile. 
Al termine di queste operazioni la campagna può essere avviata.

Sarà poi compito del cronjob di avvio, che parte ogni 30 minuti, controllare 
tutte le campagne attive e far partire l'aggiornamento automatico per i server 
schedulati nelle varie fasce orario.

\begin{figure}[H]
  \begin{flushright}
    \centering
    \includegraphics[width=0.98\textwidth]{imgs/hotfix_activity_diagram.pdf}
    \caption{Diagramma delle attività per la creazione di una campagna}
    \label{fig:Diagramma delle attività per la creazione di una campagna}
  \end{flushright}
\end{figure}
  %%%%%%%%%% CAPITOLO DI TESI %%%%%%%%%%
%
% Capitolo "3" Capitolo 3
%
%%%%%%%%%%%%%%%%%%%%%%%%%%%%%%%%%%%%%%
\mcchap{Progettazione}{cap:cap3}
\section{Entità ed associazioni}
Nella fase di progettazione concettuale del tool, uno dei principali
obiettivi è stata la definizione di un modello di dati che potesse 
supportare le funzionalità principali dell'applicazione.
A tal fine, si è deciso di realizzare un database relazionale che 
contenesse diverse tabelle. La creazione delle tabelle è gestita 
tramite Django, un framework Python molto conosciuto nello sviluppo di 
applicativi backend, verrà dettagliato in seguito.\\

Il database relazionale comprende quindi diverse tabelle, tra cui:
\begin{itemize}
\item campaign: la tabella principale del database, che rappresenta una 
campagna di aggiornamento. Ogni record in questa tabella contiene 
informazioni sulle specifiche della campagna;
\item item: tabella che contiene l'insieme dei server che devono essere 
aggiornati, correlati alla campagna tramite una relazione uno-a-molti. 
Ogni record in questa tabella rappresenta un singolo server da aggiornare;
\item updatefile: tabella che rappresenta i file di aggiornamento 
effettivi che verranno installati sui server selezionati. 
Anche questa tabella è correlata alla tabella "campaign" tramite una 
relazione uno-a-molti e contiene un record per ogni diverso sistema 
operativo dei server selezionati in una campagna di aggiornamento;
\item updatejob: tabella che contiene le informazioni sulle sessioni 
di aggiornamento avviate su un server. Ogni record in questa tabella 
rappresenta un'operazione di aggiornamento. È possibile avere più operazioni 
di aggiornamento per un singolo server, per consentire di ripetere 
l’aggiornamento in caso di fallimento;
\item updatejobstatushistory: tabella che tiene traccia dello stato di 
avanzamento dell’operazione di aggiornamento. Questa tabella è utilizzata 
per tenere traccia degli stati di ogni singola operazione di aggiornamento 
presente nella tabella.
\end{itemize}

\begin{figure}[H]
  \begin{flushright}
    \centering
    \includegraphics[width=0.90\textwidth]{imgs/ER_schema.png}
    \caption{Schema entità-relazione database relazionale}
    \label{fig:Schema entità-relazione database relazionale}
  \end{flushright}
\end{figure}

Come mostra l’immagine dello schema ER, le tabelle del database sono state 
progettate in modo da riflettere le relazioni tra le entità coinvolte 
nel sistema.

La tabella "item" è correlata alla tabella "campaign" attraverso una 
relazione uno-a-molti, in quanto ad ogni campagna di aggiornamento 
corrisponde un insieme di server che devono essere aggiornati.

La tabella "updatefile" è sempre correlata alla tabella "campaign" 
attraverso una relazione uno-a-molti, in quanto ad ogni campagna di 
aggiornamento corrisponde un insieme di file di aggiornamento che 
devono essere installati sui server selezionati.

La tabella "updatejob" è correlata alla tabella "item" in quanto ogni 
operazione di aggiornamento è associata ad un singolo server.

La tabella "updatefile" è associata alla tabella “campaign”, sempre 
attraverso relazioni uno-a-molti, perché in base al sistema operativo 
ogni server della campagna avrà il suo file di aggiornamento. 

Infine, la tabella "updatejobstatushistory" è correlata alla tabella 
"updatejob" tramite una relazione uno-a-molti, in quanto ad ogni 
operazione di aggiornamento corrisponde un insieme di stati di 
avanzamento dell'operazione.

Inoltre, nello schema, è presente la tabella auth\_user. 
Questa tabella è utilizzata da Django per gestire le utenze che 
utilizzano il backend. Infatti, ogni campagna creata è associata ad 
un creatore e ogni file di aggiornamento caricato è associato ad un 
utente che l’ha caricato.

\subsection{Definizioni delle tabelle}

\textbf{Campaign:}
\begin{itemize}
\item id: l'ID univoco della campagna;
\item name: il nome associato alla campagna;
\item status: lo stato della campagna, può essere draft (bozza), scheduled (pianificata), started (avviata) o closed (chiusa);
\item type: il tipo della campagna, può essere pkg (pacchetto) o cve;
\item mode: la modalità della campagna, può essere standard o zero day;
\item id\_ticket: l'ID del ticket associato alla campagna;
\item start\_at: la data e l'ora di inizio della campagna;
\item end\_at: la data e l'ora di fine della campagna;
\item idrs: l’ID del cliente a cui è associata la campagna;
\item created\_by: l'utente che ha creato la campagna;
\item algorithm\_started\_at: la data e l'ora di inizio dell'algoritmo di elaborazione;
\item algorithm\_scheduling\_note: una nota di pianificazione dell'algoritmo di elaborazione;
\item created\_at: la data e l'ora di creazione della campagna;
\item updated\_at: la data e l'ora di ultima modifica della campagna.
\end{itemize}

\textbf{Item:}
\begin{itemize}
\item id: l'ID univoco dell’item;
\item campaign\_id: l’ID della campagna associata all'elemento;
\item atlantis\_id: l'ID del gestionale CMDB sul quale sono salvate le informazioni del server;
\item scheduled\_at: la data e l'ora di pianificazione dell'aggiornamento;
\item has\_winrm: indica se il server ha WinRM abilitato, attraverso un check eseguito alla creazione della campagna;
\item has\_credential: indica se il server ha una credenziale associata, attraverso un check eseguito alla creazione della campagna;
\item approved: indica se l'elemento è stato approvato automaticamente (se need\_approve è False);
\item need\_approve: indica se l'elemento richiede approvazione;
\item manually\_update: indica se l'elemento deve essere aggiornato manualmente;
\item approved\_at: la data e l'ora di approvazione dell'elemento;
\item approved\_by: l'utente che ha approvato l'elemento.
\end{itemize}

\textbf{UpdateFile:}
\begin{itemize}
\item id: l'ID del file.
\item campaign\_id: l'ID della campagna associata all'aggiornamento;
\item os\_name: il codice del sistema operativo associato all'aggiornamento;
\item kb\_code: il codice univoco dell'aggiornamento di sicurezza;
\item execute\_params: i parametri di esecuzione dell'eseguibile di aggiornamento;
\item path: l'url del percorso in cui è stato caricato il file di aggiornamento, caricato su S3;
\item updated\_by: l'utente che ha caricato il file di aggiornamento;
\item s3\_version: la versione di S3 del file di aggiornamento;
\item created\_at: la data e l'ora di creazione del file;
\item updated\_at: la data e l'ora di ultima modifica del file.
\end{itemize}

\textbf{UpdateJob:}
\begin{itemize}
\item id: campo UUID che rappresenta la chiave univoca della sessione di aggiornamento;
\item item\_id: l'ID che si riferisce al modello Item e specifica l'elemento di cui viene effettuato l'aggiornamento;
\item provision\_id: contiene l'ID del gestore dei task schedulati remoti, utilizzato per far partire l'aggiornamento da remoto;
\item token\_sha256: contiene il valore SHA256 del token utilizzato per inviare aggiornamenti di stato da remoto;
\item ticket\_id: campo che può contenere l'ID del ticket che viene aperto in caso di problemi di aggiornamento automatico;
\item check\_status: campo che contiene lo stato di tutti i controlli di monitoraggio, prima di avviare l'aggiornamento;
\item silencer\_id: l'ID del downtime associato al server per evitare segnalazioni dovute al riavvio della macchina durante l'aggiornamento;
\item created\_at: la data e l'ora dell'avvio dell'aggiornamento.
\end{itemize}

\textbf{UpdateJobStatusHistory:}
\begin{itemize}
\item update\_job\_id:  l'ID della sessione di aggiornamento;
\item status: specifica lo stato dell'aggiornamento. I possibili valori sono: "Ready", "Execute error from provision", "Waiting for ping", "Timeout ping", "Waiting for download process", "Timeout download", "Running", "Timeout running", "Rebooting", "Failed", "Completed";
\item created\_at: la data e l'ora del passaggio di stato;
\item note: campo che contiene eventuali note aggiuntive per lo stato dell'aggiornamento.
\end{itemize}

  %%%%%%%%%% CAPITOLO DI TESI %%%%%%%%%%
%
% Capitolo "3" algoritmo di schedulazione
%
%%%%%%%%%%%%%%%%%%%%%%%%%%%%%%%%%%%%%%
\section{Algoritmo di schedulazione}
L’Hotfix Response Center è un tool che verrà utilizzato in momenti 
critici, per aggiornare da remoto centinaia di server vulnerabili, 
in modo automatico.

Ricopre quindi un ruolo molto importante l’algoritmo di schedulazione. 
L’algoritmo dovrà essere in grado di trovare una schedulazione 
migliore per aggiornare i server tenendo conto, durante l'elaborazione, 
di numerosi fattori, dettagliati in seguito. L’affidabilità e la 
precisione dell’algoritmo sarà fondamentale per poter garantire che 
tutto avvenga in modo automatico.

La schedulazione avverrà generalmente su un periodo temporale 
di 2/3 settimane. Ogni giorno della settimana verrà suddiviso in 
slot da 30 minuti. Per ogni slot di tempo sarà possibile aggiornare 
massimo 5 server contemporaneamente. Questo rende possibile aggiornare 
fino a 240 server al giorno. La scelta di posizionare più server nello 
stesso slot dovrà avvenire solamente nel caso non ci siano altri slot 
liberi, preferendo sempre slot vuoti. In una situazione ideale si schedula 
solo 1 server per slot, questo per garantire uniformità nel calendario.\\

\textbf{Dati in input:} data e ora di inizio e fine campagna, elenco dei 
server da schedulare dove per ogni server sono presenti:
\begin{itemize}
\item atlantis\_id: id del gestionale dal quale provengono le informazioni del server;
\item fascia di manutenzione/aggiornamento;
\item gruppo di patch: duo o più server con lo stesso gruppo di patch non possono 
essere schedulati insieme;
\item dipendenze tra server. Questi server devono essere i primi ad essere 
schedulati a distanza di 24 ore.
\end{itemize}

\textbf{Dati in output:} schedulazione per ogni server in modo da poterli 
aggiornare il prima possibile.


\subsection{Concetto di euristica}
L’algoritmo realizzato per la schedulazione si basa sul principio di euristica. 
Un'euristica è una tecnica o un metodo basato sull'esperienza o sulla 
conoscenza generale del problema, che cerca di trovare una soluzione 
ragionevole in tempi ragionevoli, anche in presenza di problemi di 
complessità elevata. In altre parole, un'euristica è un approccio basato 
sul buon senso, sulla logica e sulla creatività che, seppur non garantisca la 
soluzione ottimale del problema, può fornire risultati accettabili in tempi 
ragionevoli. Le euristiche sono spesso utilizzate per risolvere problemi di 
ottimizzazione in cui il tempo di calcolo richiesto per trovare una soluzione 
ottimale è troppo elevato.

Nel caso in questione l’algoritmo di schedulazione non esplora tutti i possibili 
casi per trovare la soluzione migliore ma si basa su dei calcoli preventivi. 
Successivamente verranno utilizzati per raggiungere, con un tempo di esecuzione 
lineare, quella che è una soluzione ottimale per creare un calendario di 
schedulazione per l’aggiornamento.
L’euristica si baserà sul calcolo dell’“availability score” che verrà utilizzato 
per decidere con che ordine schedulare i server all’interno del calendario.

È appropriato utilizzare un’euristica in quanto il calcolo della soluzione 
ottimale, che controllerebbe tutte le possibili schedulazioni, richiederebbe un 
tempo di esecuzione quadratico. Il calcolo è l’utilizzo del “availability score” 
per la schedulazione richiederebbe una complessità lineare in termini di tempo e 
spazio che fornirebbe una soluzione accettabile nel minor tempo possibile.


\subsection{Fascia di manutenzione}
Anche chiamata fascia di aggiornamento se presente indica le fasce orarie 
in cui il server può essere aggiornato e riavviato. 
Ad esempio: MAR-MER-GIO(13:00-18:00) indica che un server può essere 
aggiornato martedì, mercoledì e giovedì dalle 13 alle 18.
Se la fascia di manutenzione non è presente il server può sempre essere aggiornato.

La fascia di manutenzione è rappresentata da una stringa che deve essere 
interpretata correttamente prima di poter eseguire l’algoritmo di schedulazione. 

I giorni della settimana sono rappresentati dalle stringhe: 
LUN, MAR, MER, GIO, VEN, SAB, DOM. Seguiti dalla fascia oraria tra parentesi.
Una fascia di manutenzione può anche essere composta da più blocchi di fasce 
orarie, concatenati dal carattere +, come in questo esempio: 
LUN-MAR-MER(20:00-24:00)+LUN-MAR-MER(00:00-05:00)+SAB-DOM(00:00-24:00)

Per poter interpretare correttamente e capire durante la settimana in quali 
slot di 30 minuti può essere inserito il server è stata creata una classe 
Python capace di interpretare la stringa.

Le funzionalità principali della classe sono:
\begin{itemize}
\item Interpretare la stringa della fascia di manutenzione;
\item Gestire una struttura dati per memorizzare in quali slot durante la 
settimana il server può essere in manutenzione;
\item Esporre un metodo che comunica se il server è in manutenzione in un 
determinato slot. Questo metodo verrà poi utilizzato dall’algoritmo di schedulazione.
\end{itemize}

La classe interpreta la stringa utilizzando delle regex (abbreviazione di 
"regular expressions", ovvero "espressioni regolari" in italiano) in grado di 
riconoscere pattern di testo ed eseguire successivamente gli opportuni calcoli.
La struttura dati per memorizzare la disponibilità degli slot si basa su un 
dizionario in cui la chiave del dizionario è la rappresentazione a stringa 
dell’inizio dello slot in secondi.

\textbf{Esempio:}\\
Chiave 0 = primo slot, lunedì mattina alle ore 00:00\\
Chiave 1800 = secondo slot, lunedì mattina alle ore 00:30\\
Chiave 86400 = slot del martedì alle ore 00:00\\

Questi valori sono ottenuti impostando la costante RESOLUTION\_IN\_MINUTE = 30. 
Quindi durante la settimana, nel dizionario, saranno presenti 336 slot da 30 minuti.

Dopo aver inizializzato la struttura dati per salvare le disponibilità negli 
slot inizia l’interpretazione della stringa della fascia di manutenzione.\\

Di seguito è riportato il metodo calc\_available\_slots presente all'interno della classe MaintenanceWindowParser:

\lstinputlisting[style=custompython, language=Python]{code/calc_available_slot.py}

Al termine dell’esecuzione il dizionario contenuto in self.data conterrà 336 chiavi. 
Ad ogni chiave è associato un valore booleano. Se il valore corrisponde a True il 
server è disponibile in quello slot per l’aggiornamento.


\subsection{Calcolo dello score}
Dopo aver trovato per ogni server quali sono gli slot disponibili durante la 
settimana si procede con il calcolo dell’“availability score” di ogni server. 
Questo score sarà utilizzato per decidere quali server andranno schedulati per primi.
Availability score corrisponde al numero di secondi in cui il server è disponibile 
per l’aggiornamento durante la settimana.
Più lo score sarà basso e meno slot ha un server a disposizione per essere aggiornato.\\

Attraverso l’approccio euristico è stato constatato che schedulando per primi i 
server con un availability score più basso non compromette la schedulazione di 
altri server. Non risulterà quindi necessario provare ogni possibile schedulazione 
per trovare la migliore ma ci si potrà accontentare della schedulazione deterministica 
che verrà creata assegnando prima i server con score più basso negli slot in cui 
possono essere aggiornati.\\

Di seguito è riportato il metodo get\_num\_of\_seconds\_of\_available\_slot della 
classe MaintenanceWindowParser:

\lstinputlisting[style=custompython, language=Python]{code/get_num_of_seconds_of_available_slot.py}


\subsection{Albero delle dipendenze}
Prima dell’esecuzione dell’algoritmo di schedulazione dovrà essere creato 
l’albero delle dipendenze, per i server con dipendenze, cioè che necessitano 
che un server venga aggiornato prima di eseguire l’aggiornamento del server 
stesso. Questi server saranno schedulati per primi, tenendo una distanza di 24 
ore tra uno slot e l’altro.

La struttura utilizzata è un albero. Alla radice è presente un nodo padre. 
I figli di primo livello sono tutti i server che non hanno dipendenze ma che 
fanno dipendere altri server, e così via…

L’albero viene gestito da una semplice classe Python, che viene popolata durante 
lo scorrimento di tutti i server.


\subsection{Esecuzione dell’algoritmo}
TESTO DA RIFARE!!!!
viene creato un calendario vuoto utilizzando la classe HotfixCalendar, in cui 
l'intervallo di tempo del calendario viene impostato come la differenza tra la
 data di inizio e di fine della campagna, con una durata degli slot e un numero 
 massimo di elementi per slot specificati dall'utente.

Il metodo quindi calcola l'albero delle dipendenze per assegnare priorità alle 
attività di aggiornamento che hanno dipendenze dalle altre attività. 
Vengono quindi calcolati i punteggi di priorità per ogni attività di aggiornamento 
in base alla profondità dell'attività nell'albero delle dipendenze e alla 
disponibilità della finestra di manutenzione.

Il metodo cerca di trovare uno slot disponibile per ogni attività di aggiornamento 
in base al punteggio di priorità dell'attività e alla disponibilità degli slot. 
Se l'attività ha dipendenze, viene programmata solo dopo che tutte le attività 
dipendenti sono state programmate.

Codice parziale dell'algoritmo di schedulazione:
\lstinputlisting[style=custompython, language=Python]{code/scheduling.py}

TOGLIERE CODICE GOOGLE DOCS ALBERO DIPENDENZE

CALCOLARE COMPLESSITA'

ESEMPIO TEMPORALI DI ESECUZIONE

  \mcchap{Implementazione}{cap:cap4}
\section{Boh}

  \mcchap{Conclusioni}{cap:cap5}

\section{Soluzioni già presenti sul mercato}
Big Fix di IBM

\section{Sviluppi futuri}
Il futuro discovery automatico delle CVE e analisi dei software installati sulle macchine.


  \appendix
  %%\chapter*{Ringraziamenti}
\addcontentsline{toc}{chapter}{Ringraziamenti} % add it to the table of contents

\noindent Ringrazio i miei genitori per avermi cresciuto e per avermi costantemente supportato.
Ringrazio mio fratello Daniel, i miei nonni, gli zii e mio cugino per esserci sempre stati.\\

\noindent Ringrazio i miei compagni di corso con i quali ho condiviso questi anni di studio.
Ringrazio particolarmente Manuel per avermi tenuto sul pezzo anche quando non potevo 
seguire le lezioni universitarie.\\

\noindent Ringrazio i miei amici e le mie amiche che ho conosciuto in questi anni e 
con cui ho condiviso esperienze.\\

\noindent Ringrazio tutti i colleghi di Elmec Informatica S.p.A. con il quale ho 
lavorato negli ultimi quattro anni e che mi hanno aiutato a crescere professionalmente. 
Molto di quello che stato realizzato in questo progetto di tesi 
è stato fatto applicando le conoscenze acquisite lavorando insieme a loro.\\

\noindent Ringrazio il mio relatore nonché professore per la disponibilità e il 
rapido supporto durante la stesura di questa tesi.\\


  
  \backmatter{}

  %% Bibliografia
  \bibliography{biblio/biblio} 
  \bibliographystyle{ieeetr}

\end{document}
