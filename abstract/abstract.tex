\chapter*{Sommario}
Nell'era digitale in cui viviamo, i software che regolano le nostre 
infrastrutture e le reti tecnologiche sono sempre più esposte a minacce 
e attacchi informatici. Tali attacchi
possono essere messi in atto sfruttando vulnerabilità presenti nei sistemi 
informatici, mettendo a repentaglio la sicurezza delle aziende e/o delle 
infrastrutture nazionali.
Nel mio progetto di tesi, ho approfondito il tema delle vulnerabilità nei 
software, sviluppando uno strumento semi-automatico in grado di gestire 
l’aggiornamento tempestivo dei server affetti da vulnerabilità, per poter 
mettere in sicurezza i sistemi informatici nel più breve tempo possibile.
\newline

Il progetto realizzato ha l'obiettivo di gestire l’aggiornamento straordinario 
di server Linux e Windows, impattati da vulnerabilità critiche. Lo strumento 
sarà utilizzato creando campagne di aggiornamento che possono includere 
centinaia di server dei clienti, impattati dalla vulnerabilità. 
L'obiettivo del progetto è trovare una schedulazione ottimale per aggiornare 
tutti i server della campagna, connettersi da remoto ai server e installare 
in modo autonomo l'aggiornamento rilasciato dal produttore del software.

Le campagne di aggiornamento hanno una durata limitata in cui tutti i server 
devono avere a disposizione due ore per essere aggiornati. Per trovare una 
schedulazione ottimale l’algoritmo realizzato deve tenere in considerazione 
diversi fattori, il più importante è la fascia di manutenzione. 
Per ogni server è definito un periodo durante la settimana in cui può essere 
fuori servizio, per attività di manutenzione.
\newline

L’algoritmo sviluppato per la generazione del calendario di aggiornamento non 
è stato sviluppato per trovare la soluzione ottimale, andando a provare ogni 
singola combinazione di schedulazione, ma utilizza un approccio euristico per 
garantire in un tempo lineare una schedulazione efficiente che è molto simile 
alla soluzione ottimale.
L’approccio euristico utilizza la conoscenza di un problema per trovare la 
soluzione a problemi complessi in tempi ragionevoli. Tramite questo approccio, 
prima di cercare la schedulazione dei server, viene calcolato un punteggio per 
ogni server, che dipende da vari fattori (tra cui il numero di ore in cui il 
server è disponibile durante la settimana per la manutenzione). Successivamente, 
partendo dai server con il punteggio più basso, si procede in modo lineare 
all’assegnazione dei server alle varie fasce di manutenzione, della campagna di 
aggiornamento. Questo algoritmo garantisce un notevole vantaggio in termini di 
ottimizzazione e riduzione dei tempi. Per trovare la schedulazione di 500 server 
sono necessari 0.6 secondi.

Lo strumento realizzato è indipendente da qualsiasi altro strumento ed è composto 
da un’interfaccia grafica (frontend, realizzato con il framework Vue.js), da dei 
servizi REST (backend, realizzati in Python utilizzando il framework Django), 
da un database sql, da uno storage per il caricamento dei file di aggiornamento e 
da due script necessari all’installazione dell’aggiornamento di sicurezza, 
uno specifico per Windows (scritto in Powershell) e uno per Linux (scritto in Bash).
\newline

Tramite l’interfaccia l’operatore può creare campagne di aggiornamento, monitorare 
lo stato delle campagne di aggiornamento in corso e lo stato dei singoli 
aggiornamenti sui server. Gli aggiornamenti verranno effettuati in modo automatico 
all’orario prestabilito dall’algoritmo.
\newline

In futuro è in programma lo sviluppo di ulteriori funzionalità per rendere 
automatica la creazione delle campagne di aggiornamento, effettuando controlli 
proattivi sui software installati nei server e verificando se sono affetti da 
vulnerabilità per cui è già disponibile un aggiornamento.
